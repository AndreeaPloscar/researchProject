\documentclass{article}
\usepackage{multicol}
\usepackage{geometry} 
\usepackage{indentfirst}
\geometry{a4paper,top=3cm,left=2cm,right=2cm,bottom=3cm}
\usepackage{graphicx}
\graphicspath{ {./} }
\usepackage[
backend=biber,
style=alphabetic,
sorting=ynt
]{biblatex}

\usepackage{hyperref}
\hypersetup{
    colorlinks=true,
    linkcolor=black,
    filecolor=magenta,     
    citecolor=blue,
    urlcolor=black,
    pdftitle={Overleaf Example},
    pdfpagemode=FullScreen,
    }

\addbibresource{resources.bib}


\begin{document}



\begin{titlepage}
\sloppy

\begin{center}
BABE\c S BOLYAI UNIVERSITY, CLUJ NAPOCA, ROM\^ ANIA

FACULTY OF MATHEMATICS AND COMPUTER SCIENCE

\vspace{4cm}

\Huge \textbf{Exploring the association between stroke and weather
conditions using Artificial Intelligence} 

\vspace{1cm}

\normalsize - Research Project -

\end{center}


\vspace{7cm}

\begin{flushright}
\Large{\textbf{Ploscar Andreea Alina}}
\end{flushright}

\vspace{4cm}

\begin{center}
2022-2023
\end{center}

\end{titlepage}

\pagenumbering{gobble}

\tableofcontents

\newpage

\section{Abstract}

The study presents a new approach for assessing the impact of weather conditions on stroke incidence, making use of artificial intelligence to find this correlation and predict a sudden increase in the incidence of strokes based on meteorological conditions such as differences in atmospheric pressure, temperature, atmospheric fronts.

\section{Introduction}
\normalsize Stroke was the second cause of death in Romania in 2016 \cite{Donkor:2018dg} and is a major cause of mortality worldwide. Patients hospitalized with stroke can be treated, but due the sudden appearance of stroke symptoms, these patients require immediate attention from healthcare professionals, as well as available operating rooms. This process would be more efficient if healthcare workers could be aware that the incidence of strokes in the area is expected to increase, thus they would organize the non-emergency cases and the operating rooms using this information. 
\subsection{Aim}
The aim of the study is to assess the correlation between weather conditions and stroke incidence in Transylvania, Romania. As weather conditions temperature, air pressure, atmospheric fronts, precipitations and differences in these values are taken into consideration when analysing the correlation. For a better understanding of the relationship, underlying conditions of patients are taken into consideration. As a limitation, the study does not assess the impact of personal stress factors on patients hospitalized with stroke. 

\subsection{Related Work}
\begin{itemize}

\item\large "A study of weekly and seasonal variation of stroke onset"\\

\normalsize The paper \cite{Wang:2002dg} is relevant to this study as it presents the relationship between seasons, week days, age and stroke incidents. The results show a significant weekly and seasonal variation in the occurrence of stroke and a negative dose response relationship between seasonal variations in occurrence and age. This may be caused by the significant impact that lifestyle has on the probability of a stroke, topic that will be further discussed in this study.\\

Structure:
\begin{itemize}
\item Abstract - short overview of the article, containing the problem description, the time interval, the geographical area where the study was conducted, the methods and the results
\item Introduction - presents results and limitations of existing studies and the aim of the study
\item Subjects and methods - geographical and meteorological details of the targeted area, data collection and statistical methods used
\item Results
\begin{itemize}
    \item population characteristics
    \item weekly variation of stroke occurrence and the effect of age
    \item seasonal variation of stroke occurrence and the effect of age
\end{itemize}
\item Discussion - compares the obtained results to other studies, presents conflicting results and gives possible explanations for them, conclusion
\item References - list of cited and related work
\end{itemize}
In this paper, references appear as a list in the last section, in the following format: authors (year) title journal volume:pages. When cited, the format is: (authors year).\\

Citations: 105\\
References: 73


\item\large "Weather as a Trigger of Stroke"\\

\normalsize The paper \cite{Jimenez-Condea:2008dg} analyses the relationship between daily meteorological conditions and daily as well as seasonal stroke incidence. The approach is closed to our study, as it takes into consideration daily meteorological conditions, not only seasonal ones. The results show little association between atmospheric pressure(AP) and stroke, but higher association between stroke and variations in atmospheric pressure. The variation was computed as the value of AP compared to the previous day. Our study will also take into consideration 3, 6 and 12 hours variations. 

Structure:
\begin{itemize}
\item Abstract - background, methods, results, conclusions
\item Introduction - presents results and limitations of existing studies and the aim of the study
\item Methods 
\begin{itemize}
\item Subjects - geographical, temporal and quantitative details of analysed data
\item Classification and Variables - classification of patients and types of strokes, recorded medical and meteorological data
\item Statistical Analysis - statistical methods used to analyse the data
\item Ethics - ethical guidelines
\end{itemize}
\item Results 
\begin{itemize}
    \item Descriptive Data
    \item Daily Incidence Analysis
    \item Seasonal Analysis
\end{itemize}
\item Discussion - compares the obtained results to other studies, presents strengths and limitations of the study, conclusion
\item Acknowledgements - collaborations
\item References - numbered list of cited and related work
\end{itemize}
In this paper, references appear as a numbered list in the last section, in the following format: authors title journal year;volume:pages. When cited, the format is: [number] representing index in references list.\\

Citations: 73\\
References: 34

\item\large "The association between weather conditions and stroke admissions in Turkey"\\

\normalsize \cite{Çevik:2015dg} is focused on data from turkey, area that is closer to the one covered by the present study, i.e. Romania, so the meteorological conditions will be more similar to the ones in our study. The paper takes into consideration ischemic stroke (IS), hemorrhagic stroke (HS) and subarachnoidal hemorrhage (SAH) and The results from this paper present no association between incidence of overall admissions due to strokes and meteorological parameters, but they do demonstrate an association between admissions due to SAH ans HS and weather conditions, especially temperature. 

Structure:
\begin{itemize}
\item Abstract - aim of the study, details about used data, results
\item Introduction - presents results and limitations of existing studies and the aim of the study
\item Materials and Methods 
\begin{itemize}
\item Study Design - geographical, temporal and quantitative details of analysed data, diagnosis details
\item Meteorological data - geographical and meteorological details about data
\item Statistical Analysis - statistical methods used to analyse the data
\end{itemize}
\item Results - explained results
\item Discussion - compares the obtained results to other studies, presents strengths and limitations of the study, conclusion
\item References - list of cited and related work
\end{itemize}
In this paper, references appear as a list in the last section, in the following format: authors (year) title journal volume:pages. When cited, the format is: (author year)\\

Citations: 24\\
References: 22

\item\large "An Improved Back Propagation Neural Network Model and Its Application"\\

\normalsize The article \cite{Fang:2014dg} approaches the problem of finding a relationship between stroke incidence and weather conditions using a back propagation neural network, method that is closer to the one further presented in the present study. The results are in line with the ones presented above, presenting a stronger relationship between stroke incidence and atmospheric pressure, and a weaker, negative relationship between stroke incidence and temperature.\\


Structure:
\begin{itemize}
\item Abstract - overview, aim of the study, details about used data, results
\item Introduction - background, short description about methods
\item Model imporvement
\begin{itemize}
\item Notations - legend of notations and their meanings
\item The BPNN Flow - figure of the BPNN presenting the layers, input, output
\item Forward Propagation Process of Signal - mathematical explanations for hidden and output layers
\item Back Propagation Process of Error - mathematical explanation of used formulas and functions
\item The Improved BPNN Algorithm - description of improvements
\item Other Network Parameters - list of parameters for BPNN and environment
\item Model Solution - details about training samples input, samples training, samples prediction and effect, results
\item Model evaluation and promotion - describing the model as a prediction method
\end{itemize}
\item Acknowledgments
\item References - numbered list of cited and related work 
\end{itemize}
In this paper, references appear as a list in the last section, in the following format: authors, title, journal, volume, no, pages, year. When cited, the format is: [number] representing index in references list\\

Citations: 4\\
References: 4

\item\large "Personalized Spiking Neural Network Models of Clinical and Environmental Factors to Predict Stroke"\\

\normalsize The paper \cite{Doborjeh:2022dg} proposes a new method for the identification of associations between clinical and environmental time series: spiking neural networks. This paper is relevant to the present study, as it uses machine learning methods and also takes into consideration individual and family related factors.\\

Structure:
\begin{itemize}
\item Abstract - background, purpose of the study, details about used data and methods, results
\item Introduction - medical and technical explanations, existing work and results, new approaches in this paper
\item Methods - description of method, notations, schemas, 
\begin{itemize}
\item Method and System for Personalized Predictive Modeling on Integrated Personal Clinical Data and Dynamic Data of Environmental Changes - in detail explanation of the method, charts for data visualisation, 
\item Encoding of Environmental Time‐Series Data - description of encoding method
\item Environmental Data Mapping into a Personalized SNN Model - description of data mapping, variables and dimensions
\item Unsupervised Learning in the PSNN Model - method description
\item Supervised Learning, Classification and Prediction - method description
\end{itemize}
\item Study Population and Datasets - details about involved data (quantitative, temporal, spatial, classifications by age, gender)
\item Results - explained results, charts, figures
\item Personalized Profiling of Individual Risk of Stroke Using Environmental Data
\item Discussion - compares the obtained results to other studies, presents the advancement made by the study
\item Conclusion - overview, future work
\begin{itemize}
    \item Acknowledgements
    \item Author Contribution
    \item Funding
\end{itemize}
\item Declarations - legal aspects
\begin{itemize}
    \item Ethics Approval
    \item Consent to Participate
    \item Conflict of Interest
    \item Open Access
\end{itemize}
\item References - numbered list of cited and related work
\end{itemize}
In this paper, references appear as a list in the last section, in the following format: authors, title, journal, volume, no, pages, year. When cited, the format is: [number] representing index in references list\\

Citations: 0\\
References: 0\\
Accesses: 411
\end{itemize}

\subsection{Original Contribution}

The present study adds value to the research in the medical and meteorological fields as it gives an answer to an important question: Can stroke incidence be predicted using weather conditions? The answer to this question would help healthcare professionals save more patients hospitalized with this health emergency, contributing to the decrease of mortality due to stroke. Especially in Romania, where the study is conducted, this would have a great impact on the healthcare system overall. In comparison to existing work, the study uses artificial intelligence: neural networks and clustering algorithms to find the specified correlation, is focused on a small area, Transylvania, Romania and uses data collected over nine years 2013-2021. 

\newpage
\section{Datasets}
\subsection{Meteorological Data}
Numerical data is downloaded from Meteomanz.com and is collected from meteorological stations in Transylvania. The data contains collected values by days and by hours between Jan 2013 - Dec 2021. Data preprocessing includes formatting the values to be only numerical, removing redundant data. Because stroke incidence seems to be more related to the sudden differences in temperature and atmospheric pressure rather than their absolute values, these differences were computed using data collected by hours and added to the data by days dataset as columns for each line (day). The differences were computed for every 3 hours. For this process, the Pandas library from python was used. 

\subsection{Medical Data}

The medical dataset was collected by healthcare professionals in Cluj-Napoca, Cluj, Romania between 2013-2021 and contains annonymous records of patients admitted with strokes. These records present data about the exact time of the stroke, the place where the patient was located at that time, generic data about the patient: gender, age, underlying health conditions. 


\section{Methods}
\subsection{Meteorological Data Visualization}

To visualize the preprocessed meteorological data, the following libraries were used: Bokeh, Pandas, Matplotlib. The figures presented below represent the average values for temperature and atmospheric pressure over the studied time period and the 3 hours differences in these parameters.

\begin{figure*}[h!]
\begin{multicols}{2}
\centering
\includegraphics[width=6cm]{PressureAvg.png}
\caption{Average Atmospheric Pressure}

\centering
\includegraphics[width=6cm]{TempAvg.png}
\caption{Average Temperature}

\centering
\includegraphics[width=6cm]{PressureDiff.png}
\caption{Atmospheric Pressure 3 hours Differences}

\centering
\includegraphics[width=6cm]{TempDiff.png}
\caption{Temperature 3 hours Differences}

\end{multicols}
\end{figure*}

\subsection{Medical data Clustering}

For grouping the data points representing patients, two clustering algorithms are used: K-Means Clustering and Mean-Shift Clustering. These unsupervised learning algorithms are used to identify groups of patients with similar features.

\subsubsection{K-Means Clustering}

\subsubsection{Mean-Shift Clustering}

\subsection{Identifying Variations}

Changes in the weather conditions are often times linked to atmospheric fronts passing over the area. Variations in weather parameters such as temperature and atmoshperic pressure can signal the passing of an atmoshperic front. These fronts can be classified into: cold front, warm front, stationary front and occluded front. Strokes incidence seems to be affected by variations in weather conditions, so identifying the moment a front passes over an area could lead to a prediction for stroke incidence. To verify this hypothesis, the maximum variations need to be first identified from the meteorological dataset. Two algorithms are used for this process: Feature Importance and Outliers Identification.

\subsubsection{Feature Importance}
\subsubsection{Outliers Identification}

\subsection{Neural Network}

The problem of predicting a sudden increase in the incidence of strokes  can be seen as a classification problem and a possible approach for this is using an artificial neural network. This ANN receives as input the weather conditions for a day and outputs 1 or 0, 1 if this day corresponds to an increase in stroke incidence, i.e. the conditions for this day match the previously learned model for days correlated to a larger number of patients admitted, 0 otherwise. 

\newpage
\subsection{Experiments}

To measure the performance of the algorithms and the validity of the hypothesis, multiple experiments are performed. The data is split into train, validate and test data. The train and validate data are used during the training phase of the algorithm, while the test data is used after the model is trained, to assess the obtained performance. The experiments are done using the test data. 
\\

Although the model needs to perform well, as it would be used in medical facilities and could affect the performance of the medical system, a small type I error is accepted, i.e. the model could predict a spike in the incidence of stroke, but in reality there is no spike. In this case, the medical staff would be prepared for stroke emergencies, but this does not have a negative impact on their performance. A type II error would mean that the model predicted no increase, so the doctors would not prepare for a spike that does appear in reality. This type of error should be minimized as much as possible in the model.
\\

The following metrics are used for evaluating the experiments:
\begin{itemize}
    \item Accuracy - number of correct predictions over all predictions
    \item Precision - how many positive predictions are true positives
    \item Recall - how many of the positives were detected as positive
    \item F1-Score - harmonic mean of precision and recall
\end{itemize}
Because the recall and precision should be balanced, the F1-Score is the most meaningful for the purpose of this study. The recall should be maximized, but without dropping the precision to a small value.

Value references that would validate the model:
\begin{itemize}
    \item Accuracy $>$ 90\%
    \item Precision $>$ 77\%
    \item Recall $>$ 85\%
    \item F1-Score $>$ 80\%
\end{itemize}
\newpage
\section{Ethics}
\section{Results}
\section{Discussion}
\section{Conclusion}
\newpage

\nocite{*}
\medskip
\begin{multicols}{2}
\printbibliography[heading=bibintoc]
\end{multicols}
\end{document}
